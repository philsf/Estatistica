\everymath{\displaystyle}
%\documentclass[pdftex,a4paper]{article}
\documentclass[a4paper]{article}
%%classes: article, report, book, proc, amsproc

%%%%%%%%%%%%%%%%%%%%%%%%
%% Misc
% para acertar os acentos
\usepackage[brazilian]{babel} 
%\usepackage[portuguese]{babel} 
% \usepackage[english]{babel}
% \usepackage[T1]{fontenc}
% \usepackage[latin1]{inputenc}
\usepackage[utf8]{inputenc}
\usepackage{indentfirst}
\usepackage{fullpage}
% \usepackage{graphicx} %See PDF section
\usepackage{multicol}
\setlength{\columnseprule}{0.5pt}
\setlength{\columnsep}{20pt}
%%%%%%%%%%%%%%%%%%%%%%%%
%%%%%%%%%%%%%%%%%%%%%%%%
%% PDF support

\usepackage[pdftex]{color,graphicx}
% %% Hyper-refs
\usepackage[pdftex]{hyperref} % for printing
% \usepackage[pdftex,bookmarks,colorlinks]{hyperref} % for screen

%% \newif\ifPDF
%% \ifx\pdfoutput\undefined\PDFfalse
%% \else\ifnum\pdfoutput > 0\PDFtrue
%%      \else\PDFfalse
%%      \fi
%% \fi

%% \ifPDF
%%   \usepackage[T1]{fontenc}
%%   \usepackage{aeguill}
%%   \usepackage[pdftex]{graphicx,color}
%%   \usepackage[pdftex]{hyperref}
%% \else
%%   \usepackage[T1]{fontenc}
%%   \usepackage[dvips]{graphicx}
%%   \usepackage[dvips]{hyperref}
%% \fi

%%%%%%%%%%%%%%%%%%%%%%%%


%%%%%%%%%%%%%%%%%%%%%%%%
%% Math
\usepackage{amsmath,amsfonts,amssymb}
% para usar R de Real do jeito que o povo gosta
\usepackage{amsfonts} % \mathbb
% para usar as letras frescas como L de Espaco das Transf Lineares
% \usepackage{mathrsfs} % \mathscr

% Oferecer seno e tangente em pt, com os comandos usuais.
\providecommand{\sin}{} \renewcommand{\sin}{\hspace{2pt}\mathrm{sen}}
\providecommand{\tan}{} \renewcommand{\tan}{\hspace{2pt}\mathrm{tg}}

% dt of integrals = \ud t
\newcommand{\ud}{\mathrm{\ d}}
%%%%%%%%%%%%%%%%%%%%%%%%
\date{
\bigskip
Curso: \underline{\hspace{8cm}}\\
\ \\
Turma: \underline{\hspace{1cm}} Série: \underline{\hspace{1cm}} Turno:
\underline{\hspace{1cm}}\\
\ \\
Prof: \underline{\hspace{8cm}}\\
}

\title{Probabilidade e Estatística - Trabalho 1}

\author{
{\bf Grupo K}\\
\ \\
Nome: \underline{\hspace{6cm}} RA: \underline{\hspace{2cm}} Assinatura: \underline{\hspace{4cm}}\\
Nome: \underline{\hspace{6cm}} RA: \underline{\hspace{2cm}} Assinatura: \underline{\hspace{4cm}}\\
Nome: \underline{\hspace{6cm}} RA: \underline{\hspace{2cm}} Assinatura: \underline{\hspace{4cm}}\\
Nome: \underline{\hspace{6cm}} RA: \underline{\hspace{2cm}} Assinatura: \underline{\hspace{4cm}}\\
Nome: \underline{\hspace{6cm}} RA: \underline{\hspace{2cm}} Assinatura: \underline{\hspace{4cm}}\\
Nome: \underline{\hspace{6cm}} RA: \underline{\hspace{2cm}} Assinatura: \underline{\hspace{4cm}}\\
Nome: \underline{\hspace{6cm}} RA: \underline{\hspace{2cm}} Assinatura: \underline{\hspace{4cm}}\\
Nome: \underline{\hspace{6cm}} RA: \underline{\hspace{2cm}} Assinatura: \underline{\hspace{4cm}}\\
Nome: \underline{\hspace{6cm}} RA: \underline{\hspace{2cm}} Assinatura: \underline{\hspace{4cm}}\\
Nome: \underline{\hspace{6cm}} RA: \underline{\hspace{2cm}} Assinatura: \underline{\hspace{4cm}}\\
}

% Usar tabela do RStudio
\usepackage{longtable,booktabs}

\begin{document}
\maketitle
\newpage

%%%%%%%%%%%%%%%%%%%%%%%%
%% Título e cabeçalho
%\noindent\parbox[c]{.15\textwidth}{\includegraphics[width=.15\textwidth]{logo}}\hfill
\parbox[c]{.825\textwidth}{\raggedright%
  \sffamily {\LARGE

Probabilidade e Estatística: Trabalho 1

\par\bigskip}
{Prof: Felipe Figueiredo\par}
{\url{http://sites.google.com/site/proffelipefigueiredo}\par}
}

Versão: \verb|20160315|

%%%%%%%%%%%%%%%%%%%%%%%%


%%%%%%%%%%%%%%%%%%%%%%%%

\section{Conteúdo}

Seção {\bf Juntando tudo}, do capítulo 2 (pgs 98 e 99) do livro texto da disciplina (PLT). 

\subsection{Dados}
Seu grupo é o {\bf Grupo K}.

Seu grupo deve responder às questões do trabalho, mas não deve usar os dados do livro. Os dados que seu grupo deve usar, são os abaixo:

\begin{longtable}[c]{@{}cccc@{}}
\toprule
\begin{minipage}[b]{0.06\columnwidth}\centering\strut
A
\strut\end{minipage} &
\begin{minipage}[b]{0.06\columnwidth}\centering\strut
B
\strut\end{minipage} &
\begin{minipage}[b]{0.06\columnwidth}\centering\strut
C
\strut\end{minipage} &
\begin{minipage}[b]{0.06\columnwidth}\centering\strut
D
\strut\end{minipage}\tabularnewline
\midrule
\endhead
\begin{minipage}[t]{0.06\columnwidth}\centering\strut
2450
\strut\end{minipage} &
\begin{minipage}[t]{0.06\columnwidth}\centering\strut
1283
\strut\end{minipage} &
\begin{minipage}[t]{0.06\columnwidth}\centering\strut
1915
\strut\end{minipage} &
\begin{minipage}[t]{0.06\columnwidth}\centering\strut
1438
\strut\end{minipage}\tabularnewline
\begin{minipage}[t]{0.06\columnwidth}\centering\strut
1954
\strut\end{minipage} &
\begin{minipage}[t]{0.06\columnwidth}\centering\strut
2261
\strut\end{minipage} &
\begin{minipage}[t]{0.06\columnwidth}\centering\strut
1780
\strut\end{minipage} &
\begin{minipage}[t]{0.06\columnwidth}\centering\strut
2675
\strut\end{minipage}\tabularnewline
\begin{minipage}[t]{0.06\columnwidth}\centering\strut
1890
\strut\end{minipage} &
\begin{minipage}[t]{0.06\columnwidth}\centering\strut
1838
\strut\end{minipage} &
\begin{minipage}[t]{0.06\columnwidth}\centering\strut
2140
\strut\end{minipage} &
\begin{minipage}[t]{0.06\columnwidth}\centering\strut
1654
\strut\end{minipage}\tabularnewline
\begin{minipage}[t]{0.06\columnwidth}\centering\strut
2089
\strut\end{minipage} &
\begin{minipage}[t]{0.06\columnwidth}\centering\strut
1841
\strut\end{minipage} &
\begin{minipage}[t]{0.06\columnwidth}\centering\strut
1540
\strut\end{minipage} &
\begin{minipage}[t]{0.06\columnwidth}\centering\strut
2184
\strut\end{minipage}\tabularnewline
\begin{minipage}[t]{0.06\columnwidth}\centering\strut
2390
\strut\end{minipage} &
\begin{minipage}[t]{0.06\columnwidth}\centering\strut
2391
\strut\end{minipage} &
\begin{minipage}[t]{0.06\columnwidth}\centering\strut
1373
\strut\end{minipage} &
\begin{minipage}[t]{0.06\columnwidth}\centering\strut
2016
\strut\end{minipage}\tabularnewline
\begin{minipage}[t]{0.06\columnwidth}\centering\strut
2043
\strut\end{minipage} &
\begin{minipage}[t]{0.06\columnwidth}\centering\strut
2569
\strut\end{minipage} &
\begin{minipage}[t]{0.06\columnwidth}\centering\strut
1555
\strut\end{minipage} &
\begin{minipage}[t]{0.06\columnwidth}\centering\strut
2133
\strut\end{minipage}\tabularnewline
\begin{minipage}[t]{0.06\columnwidth}\centering\strut
2207
\strut\end{minipage} &
\begin{minipage}[t]{0.06\columnwidth}\centering\strut
1939
\strut\end{minipage} &
\begin{minipage}[t]{0.06\columnwidth}\centering\strut
1424
\strut\end{minipage} &
\begin{minipage}[t]{0.06\columnwidth}\centering\strut
1385
\strut\end{minipage}\tabularnewline
\begin{minipage}[t]{0.06\columnwidth}\centering\strut
1096
\strut\end{minipage} &
\begin{minipage}[t]{0.06\columnwidth}\centering\strut
1695
\strut\end{minipage} &
\begin{minipage}[t]{0.06\columnwidth}\centering\strut
1733
\strut\end{minipage} &
\begin{minipage}[t]{0.06\columnwidth}\centering\strut
2559
\strut\end{minipage}\tabularnewline
\begin{minipage}[t]{0.06\columnwidth}\centering\strut
2178
\strut\end{minipage} &
\begin{minipage}[t]{0.06\columnwidth}\centering\strut
1719
\strut\end{minipage} &
\begin{minipage}[t]{0.06\columnwidth}\centering\strut
1829
\strut\end{minipage} &
\begin{minipage}[t]{0.06\columnwidth}\centering\strut
1472
\strut\end{minipage}\tabularnewline
\begin{minipage}[t]{0.06\columnwidth}\centering\strut
1592
\strut\end{minipage} &
\begin{minipage}[t]{0.06\columnwidth}\centering\strut
2384
\strut\end{minipage} &
\begin{minipage}[t]{0.06\columnwidth}\centering\strut
1826
\strut\end{minipage} &
\begin{minipage}[t]{0.06\columnwidth}\centering\strut
2103
\strut\end{minipage}\tabularnewline
\bottomrule
\end{longtable}

\section{Valor}
O trabalho valerá até $3.0$pts que serão somados com a nota da P1, compondo assim a nota do primeiro bimestre (B1).

\section{Entrega}

O Trabalho 1 deverá ser entregue na aula e horário normais da semana indicada no endereço abaixo:

\url{https://sites.google.com/site/proffelipefigueiredo/anhanguera/2016}.

\subsection{Observações}

\begin{enumerate}
\item Usar o conjunto de dados designado para seu grupo, e não os dados do livro.
\item Este trabalho pode ser feito a lápis, ou a caneta. Respostas finais devem estar a caneta, e destacadas. 
\item Caso prefira, o grupo pode também fazer o trabalho no computador e entregar impresso.
\item O trabalho deve ser entregue com a capa providenciada. Esta folha de instruções não é necessária.
\item A assinatura de cada integrante é {\bf obrigatória} para que este receba a nota. Sem assinatura na capa, não adianta chorar.
\end{enumerate}

\end{document}
