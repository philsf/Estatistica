\everymath{\displaystyle}
%\documentclass[pdftex,a4paper]{article}
\documentclass[a4paper]{article}
%%classes: article, report, book, proc, amsproc

%%%%%%%%%%%%%%%%%%%%%%%%
%% Misc
% para acertar os acentos
\usepackage[brazilian]{babel} 
%\usepackage[portuguese]{babel} 
% \usepackage[english]{babel}
% \usepackage[T1]{fontenc}
% \usepackage[latin1]{inputenc}
\usepackage[utf8]{inputenc}
\usepackage{indentfirst}
\usepackage{fullpage}
% \usepackage{graphicx} %See PDF section
\usepackage{multicol}
\setlength{\columnseprule}{0.5pt}
\setlength{\columnsep}{20pt}
%%%%%%%%%%%%%%%%%%%%%%%%
%%%%%%%%%%%%%%%%%%%%%%%%
%% PDF support

\usepackage[pdftex]{color,graphicx}
% %% Hyper-refs
\usepackage[pdftex]{hyperref} % for printing
% \usepackage[pdftex,bookmarks,colorlinks]{hyperref} % for screen

%% \newif\ifPDF
%% \ifx\pdfoutput\undefined\PDFfalse
%% \else\ifnum\pdfoutput > 0\PDFtrue
%%      \else\PDFfalse
%%      \fi
%% \fi

%% \ifPDF
%%   \usepackage[T1]{fontenc}
%%   \usepackage{aeguill}
%%   \usepackage[pdftex]{graphicx,color}
%%   \usepackage[pdftex]{hyperref}
%% \else
%%   \usepackage[T1]{fontenc}
%%   \usepackage[dvips]{graphicx}
%%   \usepackage[dvips]{hyperref}
%% \fi

%%%%%%%%%%%%%%%%%%%%%%%%


%%%%%%%%%%%%%%%%%%%%%%%%
%% Math
\usepackage{amsmath,amsfonts,amssymb}
% para usar R de Real do jeito que o povo gosta
\usepackage{amsfonts} % \mathbb
% para usar as letras frescas como L de Espaco das Transf Lineares
% \usepackage{mathrsfs} % \mathscr

% Oferecer seno e tangente em pt, com os comandos usuais.
\providecommand{\sin}{} \renewcommand{\sin}{\hspace{2pt}\mathrm{sen}}
\providecommand{\tan}{} \renewcommand{\tan}{\hspace{2pt}\mathrm{tg}}

% dt of integrals = \ud t
\newcommand{\ud}{\mathrm{\ d}}
%%%%%%%%%%%%%%%%%%%%%%%%



\begin{document}

%%%%%%%%%%%%%%%%%%%%%%%%
%% Título e cabeçalho
%\noindent\parbox[c]{.15\textwidth}{\includegraphics[width=.15\textwidth]{logo}}\hfill
\parbox[c]{.825\textwidth}{\raggedright%
  \sffamily {\LARGE

Estatística: Lista 2

\par\bigskip}
% {Centro Universitário Anhanguera de Niterói -- UNIAN\par} 
% {Curso: Engenharia\par}
{Prof: Felipe Figueiredo\par}
{\url{http://sites.google.com/site/proffelipefigueiredo}}

\vspace{1cm}
}
%%%%%%%%%%%%%%%%%%%%%%%%


%%%%%%%%%%%%%%%%%%%%%%%%
\section{Formulário}

Probabilidade de um evento E em um espaço amostral S:

\begin{displaymath}
  P(E) = \frac{E}{S}
\end{displaymath}

Probabilidade do evento complementar

\begin{displaymath}
  P(E') = 1 - P(E)
\end{displaymath}


Probabilidade da união de eventos
\begin{displaymath}
  P(A \cup B) = P(A) + P(B) - P (A \cap B)
\end{displaymath}

Obs: Eventos mutuamente exclusivos
\begin{displaymath}
  P(A \cap B) = 0
\end{displaymath}

Probabilidade da interseção de eventos

\begin{displaymath}
  P(A \cap B) = P(A|B) \cdot P(B)
\end{displaymath}

Obs: Eventos independentes
\begin{displaymath}
  P(A|B) = P(A)
\end{displaymath}
\section{Exercícios}

\begin{enumerate}
\item Considere que os objetos (dado, baralho, etc) são
  honestos. Determine a probabilidade dos seguintes eventos:

  \begin{enumerate}
  \item obter um 4 num dado 
  \item obter um número par num dado 
  \item obter uma dama de copas num baralho 
  \item obter uma dama num baralho 
  \item obter uma carta de copas num baralho
  \item obter uma carta preta num baralho
  \item obter uma carta de figura (valete, dama ou rei) no naipe de
    espadas do baralho
  \item obter uma carta de figura num baralho
  \item obter uma carta com número par nos naipes pretos do baralho
  \end{enumerate}

\item Determine a probabilidade do evento complementar a cada um dos
  eventos do exercício anterior.

\item Considere dois eventos A e B mutuamente exclusivos. Se o evento
  A ocorre com probabilidade $\frac{1}{3}$ e o evento B ocorre com
  probabilidade $\frac{2}{5}$, qual é a probabilidade de ocorrer o
  evento A ou o evento B? 

\item Determine a probabilidade do evento complementar ao evento
  calculado no exercício anterior.

\item (Desafio) Quantos elementos tem o espaço amostral associado ao
  lançamento de dois dados simultaneamente? (Sugestão: considere pares
  ordenados, para cada possibilidade em cada dado)

\section{Problemas}

\item Você pegou um baralho e substituiu um 10 de Copas por um Ás de
  Paus. Qual é a probabilidade de observar um Ás nesse baralho
  viciado?

\item Você acrescentou um Ás de cada naipe em um baralho. Qual é
  probabilidade de observar um Ás nesse baralho viciado?

\item Tabelas em breve \ldots

% \item


% \item Uma moeda viciada tem probabilidade de $\frac{4}{5}$ de sair
%   coroa. 

% \item Um saco contém uma moeda honesta e uma moeda que tem duas
%   caras. Você retira uma moeda do saco e observa uma cara. Qual é a
%   probabilidade de ter saído a moeda honesta?

\end{enumerate}

\end{document}
