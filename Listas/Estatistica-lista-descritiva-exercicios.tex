\everymath{\displaystyle}
%\documentclass[pdftex,a4paper]{article}
\documentclass[a4paper]{article}
%%classes: article, report, book, proc, amsproc

%%%%%%%%%%%%%%%%%%%%%%%%
%% Misc
% para acertar os acentos
\usepackage[brazilian]{babel} 
%\usepackage[portuguese]{babel} 
% \usepackage[english]{babel}
% \usepackage[T1]{fontenc}
% \usepackage[latin1]{inputenc}
\usepackage[utf8]{inputenc}
\usepackage{indentfirst}
\usepackage{fullpage}
% \usepackage{graphicx} %See PDF section
\usepackage{multicol}
\setlength{\columnseprule}{0.5pt}
\setlength{\columnsep}{20pt}
%%%%%%%%%%%%%%%%%%%%%%%%
%%%%%%%%%%%%%%%%%%%%%%%%
%% PDF support

\usepackage[pdftex]{color,graphicx}
% %% Hyper-refs
\usepackage[pdftex]{hyperref} % for printing
% \usepackage[pdftex,bookmarks,colorlinks]{hyperref} % for screen

%% \newif\ifPDF
%% \ifx\pdfoutput\undefined\PDFfalse
%% \else\ifnum\pdfoutput > 0\PDFtrue
%%      \else\PDFfalse
%%      \fi
%% \fi

%% \ifPDF
%%   \usepackage[T1]{fontenc}
%%   \usepackage{aeguill}
%%   \usepackage[pdftex]{graphicx,color}
%%   \usepackage[pdftex]{hyperref}
%% \else
%%   \usepackage[T1]{fontenc}
%%   \usepackage[dvips]{graphicx}
%%   \usepackage[dvips]{hyperref}
%% \fi

%%%%%%%%%%%%%%%%%%%%%%%%


%%%%%%%%%%%%%%%%%%%%%%%%
%% Math
\usepackage{amsmath,amsfonts,amssymb}
% para usar R de Real do jeito que o povo gosta
\usepackage{amsfonts} % \mathbb
% para usar as letras frescas como L de Espaco das Transf Lineares
% \usepackage{mathrsfs} % \mathscr

% Oferecer seno e tangente em pt, com os comandos usuais.
\providecommand{\sin}{} \renewcommand{\sin}{\hspace{2pt}\mathrm{sen}}
\providecommand{\tan}{} \renewcommand{\tan}{\hspace{2pt}\mathrm{tg}}

% dt of integrals = \ud t
\newcommand{\ud}{\mathrm{\ d}}
%%%%%%%%%%%%%%%%%%%%%%%%



\begin{document}

%%%%%%%%%%%%%%%%%%%%%%%%
%% Título e cabeçalho
%\noindent\parbox[c]{.15\textwidth}{\includegraphics[width=.15\textwidth]{logo}}\hfill
\parbox[c]{.825\textwidth}{\raggedright%
  \sffamily {\LARGE

Estatística: Lista de Análise Descritiva

\par\bigskip}
{Prof: Felipe Figueiredo\par}
{\url{http://sites.google.com/site/proffelipefigueiredo}\par}
}

Versão: \verb|20141124|

%%%%%%%%%%%%%%%%%%%%%%%%


%%%%%%%%%%%%%%%%%%%%%%%%

\section{Formulário}


Amplitude dos dados:
\begin{displaymath}
  A = x_\text{max} - x_\text{min}
\end{displaymath}

Parâmetros e estatísticas:
\begin{center}
  \begin{tabular}[!h]{|c|c|c|}
    \hline
     & População ($N$) & Amostra ($n$) \\
    \hline
    Média & $\mu = \frac{\sum_{i=1}^N x_i}{N}$ & $\bar{x} = \sum_{i=1}^n
    \frac{x_i}{n}$\\
    \hline
    Variância & $\sigma^2 = \sum_{i=1}^N \frac{(x_i - \mu)^2}{N}$& $s^2 = \sum_{i=1}^n \frac{(x_i - \bar{x})^2}{n-1}$
    \\
    \hline
    Desvio padrão & $\sigma = \sqrt{\sigma^2}$& $s = \sqrt{s^2}$\\
    \hline
  \end{tabular}
\end{center}


% Média de uma população:
% \begin{displaymath}
%   \mu = \sum_{i=1}^N \frac{x_i}{N}
% \end{displaymath}

% Variância de uma população:
% \begin{displaymath}
%   \sigma^2 = \sum_{i=1}^N \frac{(x_i - \mu)^2}{N}
% \end{displaymath}

% Desvio padrão de uma população de tamanho $N$:
% \begin{displaymath}
%   \sigma = \sqrt{\sigma^2}
% \end{displaymath}

Fórmula alternativa da variância populacional:
\begin{displaymath}
  \sigma^2 = \frac{\sum x_i^2}{N} - \mu^2
\end{displaymath}

% Média de uma amostra de tamanho $n$:
% \begin{displaymath}
%   \bar{x} = \sum_{i=1}^n \frac{x_i}{n}
% \end{displaymath}

% Variância de uma amostra:
% \begin{displaymath}
%   s^2 = \sum_{i=1}^n \frac{(x_i - \bar{x})^2}{n-1}
% \end{displaymath}

% Desvio padrão de uma amostra:
% \begin{displaymath}
%   s = \sqrt{s^2}
% \end{displaymath}

Coeficiente de Variação (assumindo $\mu > 0$):
\begin{displaymath}
  CV = \frac{\sigma}{\mu}
\end{displaymath}



\section{Exercícios}

\begin{enumerate}
\item Classifique as seguintes variáveis:
  \begin{enumerate}
  \item Sexo
  \item Idade
  \item Altura (cm)
  \item Cor da pele
  \item Fumante (Sim ou Não)
  \item Número de cigarros por dia
  \item Cargo (emprego)
  \item Pressão arterial sistólica (mmHg)
  \item Pressão arterial diastólica (mmHg)
  \item Contagem de hemáceas
  \item Peso (kg)
  \item Temperatura ($^0$C)
  \item Quantidade de filhos
  \end{enumerate}

\item Calcule a média, mediana e moda (se houver) das seguintes populações:

  \begin{enumerate}
  \item $\{0\}$
  \item $\{1,2,3\}$
  \item $\{1,2,3,2\}$
  \item $\{5,5,10\}$
  \item $\{5,5,20\}$
  \item $\{5,5,-20\}$
  \item $\{-2,2,1,2,-3\}$
  \item $\{-1,2,2,1,2\}$
  \item $\{\frac{1}{2},-2,-3,\frac{1}{2},-\frac{1}{2},1,0,-\frac{1}{2}\}$
  \item $\{-\frac{1}{4},\frac{1}{2}\}$
  \item $\{\frac{1}{2},\frac{1}{3},\frac{1}{2}\}$
  \item $\{\frac{1}{2},2,\frac{1}{3},3,\frac{1}{6},6\}$
  \item $\{1,2,1,1,40\}$
  \end{enumerate}

\item Calcule a variância, desvio padrão e o coeficiente de variação
  (quando possível) das populações do exercício anterior.

\section{Problemas}


\item (Fonte: Sala dos professores) Os tamanhos de pé de alguns homens
  e mulheres foram coletados e são fornecidos abaixo.

  Homens $=\{42,40,42,41,41,40,40,42,43,42\}$

  Mulheres $=\{37,37,37,37,37,39,37,38,36,38\}$

  \begin{enumerate}
  \item Complete a tabela de frequências abaixo utilizando os dados
    como classes unitárias.

  \begin{center}
  \begin{tabular}[h]{|c|c|c|c|}
    \hline
    $x_i$ & $F_i$ & $f_i$ & Frequência acumulada\\
    \hline
    &&&\\
    \hline
    \ldots&&&\\
    \hline
  \end{tabular}
\end{center}

\item Calcule a média e o desvio padrão dos dados para homens e mulheres.

\item Calcule a média e o desvio padrão dos dados sem distinção de sexo.

\item (Desafio) Qual é a relação entre os resultados dos itens (b) e (c)?

\end{enumerate}


\item (Fonte: Sala dos professores) As alturas (em cm) de alguns
  homens e mulheres foram coletados e são fornecidos abaixo. 

  Homens $=\{178,163,170,175,175,174,171,182,190,176\}$

  Mulheres $=\{170,158,165,165,169,173,168,172,162,165\}$


  \begin{enumerate}
  \item Complete a tabela de frequências utilizando classes de alturas
    com amplitude 5cm. Defina a primeira classe começando em 160cm
    para homens, e 155cm para mulheres.

  \begin{center}
  \begin{tabular}[h]{|c|c|c|c|c|}
    \hline
    Classe & Ponto médio ($x_i$) & $F_i$ & $f_i$ & Frequência acumulada\\
    \hline
    &&&&\\
    \hline
    \ldots &&&&\\
    \hline
  \end{tabular}
\end{center}

\item Calcule a média das frequências das classes para homens e mulheres.

\item Calcule a média dos dados para homens e mulheres.

\item Calcule a média dos dados sem distinção de sexo.
\item (Desafio) Qual é a relação entre os resultados dos itens (c) e (d)?
  \end{enumerate}

\item (Fonte: IBEIA) O MEC determina se uma Instituição de Ensino
  Superior (IES) deve ser credenciada se atingir uma certa média de
  avaliação, ponderadas com pesos pré-estabelecidos. Considere as
  notas obtidas por uma determinada IES conforme os critérios
  objetivos:

  \begin{center}
    \begin{tabular}[h]{|c|c|c|}
      \hline
      Critério & Nota & Peso\\
      \hline
      Projeto Pedagógico & 1 & 0.3\\
      \hline
      Titulação do corpo docente & 3 & 0.2 \\
      \hline
      Nota no ENADE &  2 & 0.3\\
      \hline
      Estrutura &  3 & 0.2\\
      \hline
    \end{tabular}
  \end{center}

  \begin{enumerate}
  \item Calcule a média da instituição com as notas e pesos
    apresentados na tabela.
  \item Se a nota do ENADE diminuir para 1, qual será a nova média?
  \item Sabendo que se a média institucional for menor que 2, a mesma
    pode ser descredenciada pelo MEC, compare as respostas dos itens
    (a) e (b) e discuta com seus colegas o significado disso. E estude
    muito.
  \end{enumerate}
  
% \item Uma clínica oncológica registrou o tempo (em meses) entre a
%   remissão de um certo tipo de câncer e a recidiva de 48 pacientes. Os
%   dados foram ordenados e são apresentados na tabela abaixo, para
%   homens e mulheres.

%   Mulheres $=\{2, 2, 3, 3, 4, 4, 5, 5, 6, 6, 7, 7, 7, 7, 8, 8, 8,
%   8,10, 10, 11, 11, 12, 18\}$

%   Homens $=\{2, 2, 3, 4, 4, 4, 4, 7, 7, 7, 8, 9, 9, 10, 12, 15, 15,
%   15, 16, 16, 18, 18, 22, 22, 24\}$

%   \begin{enumerate}
%   \item Calcule a média, o desvio-padrão, a mediana e o coeficiente de
%     variação para cada sexo.
%   \item Faça os mesmos cálculos acima para todos os pacientes, sem
%     distinção de sexo.
%   \end{enumerate}

\item (Desafio) O que acontece com a média e com a variância se:
  \begin{enumerate}
  \item o mesmo número é somado a todos os elementos de um conjunto de dados?
  \item o mesmo número é multiplicado por todos os elementos de um
    conjunto de dados?
  \end{enumerate}

\end{enumerate}

\end{document}
