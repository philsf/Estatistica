%\documentclass[pdftex,a4paper]{article}
\documentclass[a4paper]{article}
%%classes: article, report, book, proc, amsproc

%%%%%%%%%%%%%%%%%%%%%%%%
%% Misc
% para acertar os acentos
\usepackage[brazilian]{babel} 
%\usepackage[portuguese]{babel} 
% \usepackage[english]{babel}
% \usepackage[T1]{fontenc}
% \usepackage[latin1]{inputenc}
\usepackage[utf8]{inputenc}
\usepackage{indentfirst}
\usepackage{fullpage}
% \usepackage{graphicx} %See PDF section
\usepackage{multicol}
\setlength{\columnseprule}{0.5pt}
\setlength{\columnsep}{20pt}
%%%%%%%%%%%%%%%%%%%%%%%%
%%%%%%%%%%%%%%%%%%%%%%%%
%% PDF support

\usepackage[pdftex]{color,graphicx}
% %% Hyper-refs
\usepackage[pdftex]{hyperref} % for printing
% \usepackage[pdftex,bookmarks,colorlinks]{hyperref} % for screen

%% \newif\ifPDF
%% \ifx\pdfoutput\undefined\PDFfalse
%% \else\ifnum\pdfoutput > 0\PDFtrue
%%      \else\PDFfalse
%%      \fi
%% \fi

%% \ifPDF
%%   \usepackage[T1]{fontenc}
%%   \usepackage{aeguill}
%%   \usepackage[pdftex]{graphicx,color}
%%   \usepackage[pdftex]{hyperref}
%% \else
%%   \usepackage[T1]{fontenc}
%%   \usepackage[dvips]{graphicx}
%%   \usepackage[dvips]{hyperref}
%% \fi

%%%%%%%%%%%%%%%%%%%%%%%%


%%%%%%%%%%%%%%%%%%%%%%%%
%% Math
\usepackage{amsmath,amsfonts,amssymb}
% para usar R de Real do jeito que o povo gosta
\usepackage{amsfonts} % \mathbb
% para usar as letras frescas como L de Espaco das Transf Lineares
% \usepackage{mathrsfs} % \mathscr

% Oferecer seno e tangente em pt, com os comandos usuais.
\providecommand{\sin}{} \renewcommand{\sin}{\hspace{2pt}\mathrm{sen}}
\providecommand{\tan}{} \renewcommand{\tan}{\hspace{2pt}\mathrm{tg}}

% dt of integrals = \ud t
\newcommand{\ud}{\mathrm{\ d}}
%%%%%%%%%%%%%%%%%%%%%%%%



\begin{document}

%%%%%%%%%%%%%%%%%%%%%%%%
%% Título e cabeçalho
%\noindent\parbox[c]{.15\textwidth}{\includegraphics[width=.15\textwidth]{logo}}\hfill
\parbox[c]{.825\textwidth}{\raggedright%
  \sffamily {\LARGE

Estatística: Gabarito 1

\par\bigskip}
% {Centro Universitário Anhanguera de Niterói -- UNIAN\par} 
% {Curso: Engenharia\par}
{Prof: Felipe Figueiredo\par}
{\url{http://sites.google.com/site/proffelipefigueiredo}}

\vspace{1cm}
}
%%%%%%%%%%%%%%%%%%%%%%%%


%%%%%%%%%%%%%%%%%%%%%%%%

\section{}

\section{}

\begin{enumerate}
\item 
  \begin{enumerate}
  \item Variável qualitativa nominal
  \item Variável quantitativa racional
  \item Variável quantitativa racional
  \item Variável qualitativa nominal
  \item Variável qualitativa nominal
  \item Variável quantitativa racional
  \item Variável qualitativa nominal
  \item Variável quantitativa racional
  \item Variável quantitativa racional
  \item Variável quantitativa racional
  \item Variável quantitativa racional
  \item Variável quantitativa racional
  \item Variável quantitativa racional
  \end{enumerate}

\item %Medidas de tendência central
  \begin{enumerate}
  \item $\mu=0,M_d=0,M_o=0$
  \item $\mu=2 ,M_d=2,$ não possui moda
  \item $\mu=2 ,M_d=2 ,M_o=2$
  \item $\mu=\frac{20}{3},M_d=5 ,M_o=5$
  \item $\mu=10 ,M_d=5 ,M_o=5$
  \item $\mu=-\frac{10}{3} ,M_d=5 ,M_o=5$
  \item $\mu=0 ,M_d=1 ,M_o=2$
  \item $\mu=\frac{6}{5} ,M_d=2 ,M_o=2$
  \item $\mu=-\frac{1}{2} ,M_d=-\frac{1}{4} ,M_o=-\frac{1}{2}, M_0=\frac{1}{2}$
  \item $\mu=\frac{1}{8} ,M_d=\frac{1}{8},$ não possui moda
  \item $\mu=\frac{4}{9} ,M_d=\frac{1}{2} ,M_o=\frac{1}{2}$
  \item $\mu=2 ,M_d=\frac{5}{4} ,$ não possui moda
  \item $\mu=9 ,M_d=1 ,M_o=1$
  \end{enumerate}

\item %Medidas de dispersão
  \begin{enumerate}
  \item $\sigma^2=0, \sigma=0,CV=$ não possui (divisão por zero)
  \item $\sigma^2=0.67, \sigma=0.82,CV=40.82\%$
  \item $\sigma^2=0.50, \sigma=0.71,CV=35.36\%$
  \item $\sigma^2=5.56, \sigma=2.36,CV=35.36\%$
  \item $\sigma^2=50, \sigma=7,07,CV=70.71\%$
  \item $\sigma^2=138.89, \sigma=11.79,CV=-353.55\%$
  \item $\sigma^2=4.40, \sigma=2.10,CV=$ não possui (divisão por zero)
  \item $\sigma^2=1.36, \sigma=1.17,CV=97.18\%$
  \item $\sigma^2=1.63, \sigma=1.27,CV=-254.95\%$
  \item $\sigma^2=0.14, \sigma=0.38,CV=300.00\%$
  \item $\sigma^2=0.01, \sigma=0.08,CV=17.72\%$
  \item $\sigma^2=4.23, \sigma=2.06,CV=102.85\%$
  \item $\sigma^2=240.40, \sigma=15.50,CV=172.28\%$
  \end{enumerate}

 \section{}

 \item 
   \begin{enumerate}
   \item
   \item $\mu_\text{h} = 41.30, \mu_\text{m} = 37.30$
   \item $\mu_\text{(h+m)} = 39.30$
   \item 
   \end{enumerate}

 \item 
   \begin{enumerate}
   \item
   \item (breve)
   \item $\mu_\text{h} = 175.40, \mu_\text{m} = 166.70$
   \item $\mu_\text{(h+m)} = 171.05$
   \item 
   \end{enumerate}

 \item 
   \begin{enumerate}
   \item $\bar x = 2.1$
   \item $\bar x = 1.8$
   \end{enumerate}

\end{enumerate}

\end{document}
