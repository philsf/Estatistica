\everymath{\displaystyle}
\documentclass{beamer}
% \documentclass[handout]{beamer}

%\usepackage[pdftex]{color,graphicx}
\usepackage{amsmath,amssymb,amsfonts}

\mode<presentation>
{
  % \usetheme{Darmstadt}
  % \usetheme[hideothersubsections]{Hannover}
  % \usetheme[hideothersubsections]{Goettingen}
  \usetheme[hideothersubsections, right]{Berkeley}

  \usecolortheme{seahorse}
  % \usecolortheme{dolphin}
  \usecolortheme{rose}
  % \usecolortheme{orchid}

  \useinnertheme[shadow]{rounded}

  \setbeamercovered{transparent}
  % or whatever (possibly just delete it)
}

\mode<handout>{
  \setbeamercolor{background canvas}{bg=black!5}
  \usepackage{pgfpages}
  \pgfpagesuselayout{4 on 1}[a4paper,border shrink=5mm, landscape]
}

\usepackage[brazilian]{babel}
% or whatever

% \usepackage[latin1]{inputenc}
\usepackage[utf8]{inputenc}
% or whatever

\usepackage{times}
%\usepackage[T1]{fontenc}
% Or whatever. Note that the encoding and the font should match. If T1
% does not look nice, try deleting the line with the fontenc.


\title%[] % (optional, use only with long paper titles)
{Medidas de associação}

\subtitle
{Correlação Linear} % (optional)

\author%[] % (optional, use only with lots of authors)
{Felipe Figueiredo}% \and S.~Another\inst{2}}
% - Use the \inst{?} command only if the authors have different
%   affiliation.

\institute[UNIAN] % (optional, but mostly needed)
{UNIAN - Centro Universitário Anhanguera de Niterói
}
  % \inst{1}%
  % Department of Computer Science\\
  % University of Somewhere
  % \and
  % \inst{2}%
  % Department of Theoretical Philosophy\\
  % University of Elsewhere}
% - Use the \inst command only if there are several affiliations.
% - Keep it simple, no one is interested in your street address.

\date%[] % (optional)
{}

% \subject{Talks}
% This is only inserted into the PDF information catalog. Can be left
% out. 



% If you have a file called "university-logo-filename.xxx", where xxx
% is a graphic format that can be processed by latex or pdflatex,
% resp., then you can add a logo as follows:

\pgfdeclareimage[height=1.6cm]{university-logo}{../logo}
\logo{\pgfuseimage{university-logo}}



% Delete this, if you do not want the table of contents to pop up at
% the beginning of each subsection:
\AtBeginSubsection[]
%\AtBeginSection[]
{
  \begin{frame}<beamer>{Sumário}
    \tableofcontents[currentsection,currentsubsection]
  \end{frame}
}


% If you wish to uncover everything in a step-wise fashion, uncomment
% the following command: 

\beamerdefaultoverlayspecification{<+->}


\begin{document}

\begin{frame}
  \titlepage
\end{frame}

\begin{frame}{Sumário}
  \tableofcontents
  % You might wish to add the option [pausesections]
\end{frame}


%% Template
% \section{}

% \subsection{}

% \begin{frame}{}
%   \begin{itemize}
%   \item 
%   \end{itemize}
% \end{frame}

% \begin{frame}
%   \begin{columns}
%     \begin{column}{5cm}
%     \end{column}
%     \begin{column}{5cm}
%     \end{column}
%   \end{columns}
% \end{frame}

% \begin{frame}{}
%   \includegraphics[height=0.4\textheight]{file1}
%   \includegraphics[height=0.4\textheight]{file2}
%   \includegraphics[height=0.4\textheight]{file3}
%   \begin{figure}
%     \caption{}
%   \end{figure}
% \end{frame}

% \begin{frame}{}
%   \begin{definition}
%   \end{definition}
%   \begin{example}
%   \end{example}
%   \begin{block}{Exercício}
%   \end{block}
% \end{frame}

\section{Correlação}

\subsection[Associação]{Associação entre duas variáveis}

\begin{frame}{Tipos de variáveis envolvidas}
  \begin{itemize}
  \item Considere duas amostras X e Y, de dados numéricos contínuos.
  \item Vamos representar os dados em pares ordenados (x,y) onde:
    \begin{itemize}
    \item X: variável independente (ou variável explanatória)
    \item Y: variável dependente (ou variável resposta)
    \end{itemize}
  \end{itemize}
\end{frame}

\begin{frame}{Medidas de associação}
  \begin{itemize}
  \item Como definir (e mensurar!) o grau de associação entre duas
    variáveis aleatórias (VAs)?
  \item Se uma VA é dependente de outra, é razoável assumir que isso
    possa ser observável por estatísticas sumárias
  \item Como resumir esta informação em uma única grandeza numérica?
  \end{itemize}
\end{frame}

\begin{frame}{Medidas de associação}
  \begin{itemize}
  \item Quando uma associação é forte, podemos identificá-la
    subjetivamente
  \item Para isto, analisamos o gráfico de dispersão dos pares (x,y)
  \item Um gráfico deste tipo é feito simplesmente plotando os pontos
    no plano cartesiano
  \end{itemize}
\end{frame}

\begin{frame}{Exemplo}
  \includegraphics[height=0.6\textheight]{Assoc/positive}

  (Fonte: Triola)
\end{frame}

\begin{frame}{Exemplo}
  \includegraphics[height=0.6\textheight]{Assoc/negative}

  (Fonte: Triola)
\end{frame}

\begin{frame}{Exemplo}
  \includegraphics[height=0.6\textheight]{Assoc/other}

  (Fonte: Triola)
\end{frame}

\subsection[Covariância]{Covariância entre duas amostras}

\begin{frame}{Variância}
  \begin{itemize}
  \item Relembrando: a variância (assim como o desvio-padrão) é uma
    medida da dispersão da amostra
  \item Medida sumária que resume o quanto os dados se desviam da
    média
  \item Podemos usar um raciocínio análogo para comparar quanto uma
    amostra se desvia em relação à outra
  % \item Isto é, quanto uma 
  \end{itemize}
\end{frame}

\begin{frame}{Covariância entre duas amostras}
  \begin{definition}
    A covariância entre duas variáveis X e Y é uma medida de quanto
    ambas variam juntas (uma em relação à outra).
  \end{definition}
  \begin{itemize}
  \item Obs: duas variáveis independentes tem covariância igual a zero!
  \end{itemize}
\end{frame}

\begin{frame}{Correlação}
  \begin{definition}
    A correlação é a associação estatística entre duas variáveis.
  \end{definition}
  
  Para medir essa associação, calculamos o \alert{coeficiente de
    correlação} $r$.
  
\end{frame}


% \begin{frame}{}
%   \begin{itemize}
%   \item 
%   \end{itemize}
% \end{frame}

\subsection[Pearson]{Coeficiente de correlação de Pearson}

\begin{frame}{Coeficiente de correlação}
  \begin{definition}
    O coeficiente de correlação $r$ é a medida da direção e força da
    associação entre duas variáveis.
  \end{definition}
  Propriedades:
  \begin{itemize}
  \item É um número entre $-1$ e $1$.
  \item Mede a associação \alert{linear} entre duas variáveis.
    \begin{itemize}
    \item Diretamente proporcional, inversamente proporcional, ou
      ausência de proporcionalidade.
    \end{itemize}
  \end{itemize}
\end{frame}

\begin{frame}{Coeficiente de correlação}
  \begin{itemize}
  \item O coeficiente de correlação de Pearson é a covariância
    normalizada
  \item Pode ser calculado para populações ($\rho$) ou amostras ($r$)
  \item Amostra:
    \begin{displaymath}
      r = \frac{\sum xy - n \bar{x}\bar{y}}{(n-1)s_x s_y}
%      r = \frac{n\sum xy -\left(\sum x \right)\left(\sum y \right)}{}
%\frac{\text{Cov(X,Y)}}{\sigma_X \times \sigma_Y}
    \end{displaymath}
  \item Utilizando uma fórmula semelhante, encontramos o coeficiente
    $\rho$ para uma população
  \end{itemize}
\end{frame}

\begin{frame}{Correlação}
  \begin{block}{}
    \begin{itemize}
    \item Uma forte associação \alert<2>{positiva} corresponde a uma correlação
      próxima de \alert<2>{1}.
    \item Uma forte associação \alert<3>{negativa} corresponde a uma correlação
      próxima de \alert<3>{-1}.
    \item A \alert<4>{ausência} de associação corresponde a uma
      correlação próxima de \alert<4>{0}.
    \end{itemize}
  \end{block}
\end{frame}

% \begin{frame}{}
%   \begin{itemize}
%   \item Se tivéssemos os dados de toda a população, poderíamos
%     calcular o \alert<1>{parâmetro} $\rho$
%   \item Na prática, só podemos calcular a \alert<2>{estatística} $r$ da amostra
%   \item Utilizamos $r$ como estimador para $\rho$, e testamos a
%     significância estatística da forma usual
%   \end{itemize}
% \end{frame}

\begin{frame}{Exemplo}
 \begin{example}
   Pesquisadores queriam entender por que a insulina varia tanto entre
   indivíduos. Imaginaram que a composição lipídica das células do
   músculo afetam a sensibilidade do músculo para a insulina. Para
   isto, eles injetaram insulina em 13 jovens adultos, e determinaram
   quanta glicose eles precisariam injetar nos sujeitos para manter o
   nível de glicose sanguínea constante. A quantidade de glicose
   injetada para manter o nível sanguíneo constante é, então, uma
   medida da sensibilidade à insulina.

 % Eles só precisavam injetar
   % muita glicose quando o músculo era muito sensível à insulina.

    (Fonte: Motulsky, 1995)
  \end{example}
\end{frame}

\begin{frame}{Exemplo}
  \begin{example}
    Os pesquisadores fizeram uma pequena biópsia nos músculos para
    aferir a fração de ácidos graxos poliinsaturados que tem entre 20
    e 22 carbonos (\%C20-22). Como variável resposta, mediram o índice
    de sensibilidade à insulina.
    % \begin{itemize}
    % \item A variável independente (X) é a quantidade de ácidos graxos
    % \item A variável dependente (Y) é o índice de insulina medido
    % \end{itemize}
  \end{example}
  Valores tabelados a seguir.
\end{frame}

\begin{frame}{Exemplo}
  \begin{center}
      \includegraphics[height=0.8\textheight]{Assoc/table}
  \end{center}
\end{frame}

\begin{frame}{Exemplo: Diagrama de dispersão dos dados}
  \includegraphics[width=\textwidth]{Assoc/scatter}

  Obs: na verdade, $r=0.77$.
\end{frame}

% \begin{frame}{Exemplo}
%   \begin{itemize}
%   \item O tamanho da amostra foi $n=13$
%   \item Consultamos o valor crítico de $r$ na tabela a seguir
%   \item Testamos a $H_0$ que não há relação entre as variáveis na
%     população ($H_0: \rho = 0$).
%   \end{itemize}
% \end{frame}

% \begin{frame}{Exemplo}
%   \begin{center}
%       \includegraphics[height=0.8\textheight]{Assoc/test}
%   \end{center}
% \end{frame}

% \begin{frame}{Exemplo}
%   \begin{itemize}
%   \item O valor crítico da tabela para uma amostra de tamanho 13 é
%     $r_c = 0.553$
%   \item A correlação calculada para esta amostra foi $r=0.77$
%   \item Como a correlação é maior que o valor crítico, a relação é
%     estatisticamente significativa
%   \item Conclusão: há evidências para rejeitar a $H_0$ que não há
%     relação entre as variáveis.
%   \end{itemize}
% \end{frame}

% \begin{frame}{Exemplo}
%   \begin{itemize}
%   \item Pode-se também calcular o p-valor para o coeficiente de
%     correlação $r$.
%   \item Para este exemplo, teríamos $p=0.0021$.
%   \item Interpretação: se não houver relação entre as variáveis
%     ($H_0$), existe apenas 0.21\% de chance de observamos uma
%     correlação tão forte com um estudo deste tamanho
%   \end{itemize}
% \end{frame}

\begin{frame}{Exemplo}
  Por que as duas variáveis são tão correlacionadas? Considere 4
  possibilidades:
  \begin{enumerate}
  \item o conteúdo lipídico das membranas \alert<1>{determina} a
    sensibilidade à insulina
  \item A sensibilidade à insulina de alguma forma afeta o conteúdo lipídico
  \item tanto o conteúdo lipídico quanto a sensibilidade à insulina
    estão sob o efeito de \alert<3>{algum outro} fator (talvez algum hormônio)
  \item as duas variáveis não são correlacionads na população, e a
    estimativa observada nessa amostra é mera coincidência
  \end{enumerate}
\end{frame}

% \begin{frame}{Interpretando o $r$}
%   \begin{itemize}
%   \item Nunca devemos ignorar a última possibilidade!
%   % \item o p-valor indica quão rara é essa coincidência
%   % \item neste caso, em apenas 0.21\% dos experimentos não haveria uma
%   %   correlação real, e estaríamos cometendo um erro de interpretação
%   \end{itemize}
% \end{frame}

\begin{frame}{Exercício}
  \begin{block}{Exercício}
    Dados de gastos com propaganda (x) e vendas (x), ambos em \$1000 de uma empresa.

\bigskip

\centering\begin{tabular}{|c|c|c|c|c|c|c|c|c|}
\hline
x & 2.4 &  1.6& 2.0 & 2.6 & 1.4& 1.6& 2.0& 2.2\\
\hline
y & 225& 184& 220& 240& 180& 184& 186& 215\\
\hline
    \end{tabular}

\bigskip

Qual é a correlação entre os gastos de propaganda e as vendas? O que podemos concluir deste valor?
  \end{block}

Fonte: Larson \& Farber.
\end{frame}

\begin{frame}{Exercício}
  \begin{block}{Cola}
    \begin{columns}
      \begin{column}{5cm}
        \begin{itemize}
        \item<1->  $\bar{x} = 1.975$
        \item<1->  $\bar{y} = 204.25$
        \item<1->  $s_x = 0.420034$
        \item<1->  $s_y = 23.34065$
        \end{itemize}
      \end{column}
      \begin{column}{5cm}
        \begin{itemize}
        \item<1-> $n = 8$
        \item<1-> $\sum xy = 3289.8$
        \item<1-> $r = \frac{\sum xy - n \bar{x}\bar{y}}{(n-1)s_x s_y}$
        \end{itemize}
      \end{column}
    \end{columns}
  \end{block}
  \begin{block}{Solução}
    \begin{displaymath}
      r = \frac{3289.8 - (8 \times 1.975 \times 204.25)}{7 \times 0.420034 \times 23.34065} = 0.9129053 \approx 0.913
    \end{displaymath}

Interprete!
  \end{block}
\end{frame}

\begin{frame}{Elevando o $r$ ao quadrado}
  \begin{itemize}
  \item Relembrando: calculamos a variância de uma amostra para saber
    a dispersão dos dados
  \item Sua interpretação é confusa, portanto preferimos usar o
    desvio-padrão
  \item No caso do $r$ é o contrário: a interpretação de $r^2$ é mais simples
  \item Obs: o valor $r^2$ também é chamado \alert{coeficiente de
      determinação}, como veremos a seguir.
  \end{itemize}
\end{frame}

\begin{frame}{Interpretando o $r^2$}
  \begin{itemize}
  \item No exemplo anterior, $r^2 = 0.59$
  \item no caso, 59\% da variabilidade da tolerância à insulina pode
    ser explicada pelo conteúdo lipídico
  \item Ou seja: conhecer o conteúdo lipídico permite explicar 59\%
    da variância na sensibilidade à insulina
  \item Isto deixa 41\% da variância que pode ser explicada por outros
    fatores ou erros de medição
  % \item E este valor ($r^2$) também é utilizado na Regressão!
  \end{itemize}
\end{frame}

% \subsection[IC]{Intervalo de Confiança}

% \begin{frame}{Intervalo de confiança em torno de $r$}
%   \begin{itemize}
%   \item 
%   \end{itemize}
%   \begin{example}
    
%   \end{example}
% \end{frame}

% \begin{frame}{Intervalo de confiança em torno de $r^2$}
%   \begin{itemize}
%   \item Com o IC em torno de $r$, podemos obter o IC em torno de $r^2$
%   \item Para isto, basta elevar cada extremo do IC ao quadrado
%   \end{itemize}
%   \begin{example}
    
%   \end{example}
% \end{frame}


% \begin{frame}{}
% \end{frame}

\end{document}
